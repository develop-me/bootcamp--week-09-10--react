When we make a request to the API we don't get a response straight away. Even on a fast connection it can take 100ms or more to get a response. That might not sound like a long time, but for a computer that's ages.
\\

So, JavaScript doesn't just stop doing everything until you get a response back: it keeps on doing whatever else needs to be done until it gets a response.
\\

When we don't get back a response immediately, we need to deal with the response \textbf{asynchronously}: in other words, the code that runs when the response comes back could run at \textit{any time} in the future (or never).  We've dealt with a similar concept when we wrote event handlers.


\section{Promises}

So, when we make a request with Axios, we can't be given back the response to store in a variable, as the response doesn't exist yet. What we get back instead is a \textbf{Promise}.
\\

A Promise is an object with a \texttt{.then()} method. We can pass the \texttt{.then()} method a function, which will run once the response comes back. With Axios, the function will be passed the response as the first parameter.

\begin{minted}{js}
    // make the request
    let promise = axios.get("/articles");

    // setup the handler for when the response is successful
    promise.then(response => {
        console.log(response);
    });
\end{minted}

Rather than using an intermediary variable, generally we use chaining with Promises:

\begin{minted}{js}
    // use destructuring to get the data property
    axios.get("/articles").then(({ data }) => {
        console.log(data);
    });
\end{minted}

Be aware that \textbf{there's no guarantee the promises will resolve in the order you wrote them}. For example, \texttt{GET} requests tend to be quicker than other types of request as they generally involve less server activity.
\\

We can use \href{https://httpbin.org/#/Dynamic_data/get_delay__delay_}{https://httpbin.org/delay/<delay>} to play around with this:

\begin{minted}{js}
    // will take 5 seconds to return a response
    axios.get("https://httpbin.org/delay/5").then(() => {
        // will show after (at least) 5 seconds
        console.log("Five");
    });

    // will take 1 second to return a response
    axios.get("https://httpbin.org/delay/1").then(() => {
        // will show after (at least) 1 second
        console.log("One");
    });
\end{minted}



\subsection{Errors}

Sometimes things will go wrong: you might have sent invalid data, the API server might be down, or any number of other issues.
\\

The \texttt{.then()} method of a Promise accepts a second argument which will be called if something goes wrong:

\begin{minted}{js}
    axios.get("/articles").then(response => {
        console.log("Everything has worked");
    }, error => {
        console.log(error); // logs an error message
        console.log(error.response); // the response object
        console.log("Something has gone wrong");
    });
\end{minted}

Alternatively you can use the \texttt{.catch()} method:

\begin{minted}{js}
    axios.get("/articles").then(response => {
        console.log("Everything has worked");
    }).catch(error => {
        console.log("Something has gone wrong", error.response);
    });
\end{minted}

You could use the above methods to handle form validation errors and the like.

\section{API Requests from Components}

We can put API calls in our component methods, we just need to make sure that we use \texttt{this.setState()} \textit{inside} the \texttt{.then()} function:

\begin{minted}{js}
    handleSubmit() {
        // get the values of some controlled components
        let { title, article } = this.state;

        // post data to an API
        axios.post("/api/article", {
            title: title,
            article: article,
        }).then(() => {
            // once the server gets responds successfully, clear the inputs
            this.setState({ title: "", article: "" });
        });
    }
\end{minted}

If we need to fetch data from the API data to show in our component, then we'll need to use \texttt{componentDidMount}:

\inputminted{js}{03-async/figures/01-StarWarsFolks.jsx}

If it's not doing what you expect, you can use the \href{https://developer.mozilla.org/en-US/docs/Tools/Network_Monitor}{``Network'' tab in Developer Tools} to see if your API requests are working or not.

\pagebreak

\begin{infobox}{Not Ideal}
    If we make an API request inside our component, then it means that the component can \textit{only} be used for displaying that specific data.
    \\

    However, the whole purpose of React components is that we can reuse the same bits of UI with different data. So this is far from ideal.
    \\

    When we look at Redux next week we'll see a much cleaner way to do API calls, which does not tie a component to a specific API call.
\end{infobox}

\section{Additional Resources}

\begin{itemize}[leftmargin=*]
    \item \href{https://developer.mozilla.org/en-US/docs/Web/JavaScript/Reference/Global_Objects/Promise}{MDN: Promises}
    \item \href{http://exploringjs.com/es6/ch_promises.html}{Promises: A Technical Look}
    \item \href{https://developer.mozilla.org/en-US/docs/Web/JavaScript/Reference/Statements/async_function}{MDN: \texttt{async} functions}
    \item \href{https://www.smashingmagazine.com/2019/10/asynchronous-tasks-modern-javascript/}{Writing Asynchronous Tasks In Modern JavaScript}
\end{itemize}
