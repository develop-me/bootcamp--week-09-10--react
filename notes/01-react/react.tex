\section{What is React?}

\quoteinline{React is a declarative, efficient, and flexible JavaScript library for building user interfaces}{\href{https://facebook.github.io/react/}{The React Docs}}

React is a library, created by Facebook\footnote{But it's open source, so they can't put the usual tracking/democracy-sabotaging code in there}, that makes doing DOM stuff nice. It's currently the most popular JS library for doing UIs and it looks like it may remain so.\footnote{Fingers crossed}
\\

You'll probably remember that doing stuff with the DOM was long winded, overly-complicated, and you had to be very careful about performance issues: making sure you didn't use the DOM more frequently than you needed to.
\\

React deals with all the DOM stuff for us: we just write basic components, that look almost like HTML, and it wires everything up.
\\

React is built on the philosophy that large complex apps should be built by combining small simple components.

\section{What React \textit{Isn't}}

React is \textit{not} a framework: it only deals with the rendering of HTML and DOM events. That doesn't stop a lot of people from trying to build their entire app using just React components, but that is not its intended purpose.
\\

For this reason we'll also need to use a few other libraries:

\begin{itemize}
    \item \textbf{ReactRouter}: handles page/URL changes
    \item \textbf{Redux}: looks after the \textbf{state} of our app
    \item \textbf{Axios}: lets us deal with HTTP requests
\end{itemize}

This is a common combination of libraries - an \textit{ad hoc} framework of sorts. As an added bonus, Redux and Axios are both useful even when not making React apps.
\\

Because React only deals with the UI side of things, we won't be able to build a fully working app until we've also learnt Redux. This means that we'll be a little limited in what we can built over the next two weeks: we'll be able to build lots of stand-alone components, but not an app where they all talk to each other.


\begin{infobox}{When to use React}
    React is designed for building the UIs of complex single-page web-apps: if you just need a few bits of interactivity on an otherwise non-interactive page, then you probably shouldn't use React.
    \\

    When you're thinking about using React to build something it's important to consider whether you need to use React at all:

    \begin{itemize}
        \item Using React means the user has to have JavaScript switched on for the site to do \textit{anything}
        \item Using React will add a significant amount of extra data that needs to be downloaded
        \item Using React will run slowly on older computers
    \end{itemize}

    These rules apply equally to other libraries/frameworks such as Angular, Vue, and Ember.
    \\

    As a general rule of thumb: if you have various JS bits of the page that don't need to be aware of each other (e.g. a carousel, a menu system, some form validation) then you're probably best using DOM/jQuery. If the different parts of the page \textit{do} need to know about all the others, then you're probably building an app and React would be a good choice.
\end{infobox}


\pagebreak


\section{Creating a Project}

NPM allows us to easily create the \textbf{scaffolding} for a React app (this uses Facebook's \texttt{create-react-app} package under the hood):

\begin{minted}{bash}
    npm init react-app project-name
\end{minted}

This will create a directory called \texttt{project-name} in the current working directory. You'll probably want to call your project something more descriptive.
\\

Once the directory is setup we can go into the project directory and run:

\begin{minted}{bash}
    npm start
\end{minted}

This will run a web-server on your machine, with the \texttt{public} directory as its root, and then load a web browser:

\img{12cm}{01-react/img/app}{0em}{Your shiny new app}

The page will automatically refresh when you make changes to the code. It will also tell you about any errors that you make.\footnote{This is called \textbf{linting} your code}
\\

Under the hood various packages are being used:

\begin{itemize}
    \item \textbf{Babel}: converts JSX and modern JS into browser compatible code
    \item \textbf{Webpack}: combines all the files together
    \item \textbf{Webpack Server}: serves the web page
    \item \textbf{ESLint}: checks your code for errors
\end{itemize}

One day you'll probably need to learn how to use all of these tools yourself. For now, it's best to let \texttt{create-react-app} do its thang.
\\

Let's have a look inside the newly created folder:

\begin{minted}{bash}
    .
    ├── package.json        # which packages to install
    ├── node_modules        # where npm installs packages
    ├── public              # the server root
    │   └── index.html          # the HTML template
    └── src                 # where our code lives
        ├── App.js              # the root React component
        └── index.js            # the JS entry point
\end{minted}

We'll be doing most of our work in the \texttt{src} directory.


\section{Additional Resources}

\begin{itemize}[leftmargin=*]
    \item \href{https://reactjs.org/docs/add-react-to-a-new-app.html#create-react-app}{React: Create React App}
    \item \href{https://css-tricks.com/project-need-react/}{When Does a Project Need React?}
    \item \href{https://www.seguetech.com/website-vs-web-application-whats-the-difference/}{Website vs Web Application}
    \item \href{https://jscomplete.com/learn/complete-intro-react}{The Complete Introduction to React}
    \item \href{https://medium.com/javascript-scene/the-missing-introduction-to-react-62837cb2fd76}{The Missing Introduction to React}
    \item \href{http://react.statuscode.com/}{React Status}: Weekly React Newsletter
    \item \href{https://overreacted.io}{Overreacted}: Blog about React by one of the core developers
    \item \href{https://medium.com/the-node-js-collection/modern-javascript-explained-for-dinosaurs-f695e9747b70}{Modern JavaScript Explained for Dinosaurs}
    \item \href{https://pomb.us/build-your-own-react/}{Build Your Own React}
\end{itemize}
