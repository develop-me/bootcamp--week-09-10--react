As we've seen already, a nice feature of React's one-way data flow is that it's easy to create a component that we can then use in multiple places in an app: because it doesn't need to know anything about \textit{where} it's being used we can use it \textit{anywhere}.
\\

But it doesn't even need to be the same app: we can reuse a component in \textit{any} React app. And that means that other people can write components that we can then use in our own React apps.
\\

There is a large ecosystem of pre-built React components that we can tap into. These go from individual components that do a single thing to entire component libraries.
\\

We're going to look at two examples: React Bootstrap and React Datepicker.

\begin{infobox}{Documentation}
    When we're using other people's code good documentation is essential: how else are you supposed to know the details of how something is meant to work?
    \\

    This is worth mentioning for two reasons:

    \begin{itemize}
        \item If a component doesn't have good documentation, then look elsewhere: it's not worth the effort.
        \item You will spend quite a bit of your time reading documentation. That's perfectly normal.
    \end{itemize}

    A good part of your time as a developer is spent trying to understand how to use code other people wrote. Getting good at reading documentation is therefore a key programming skill.
\end{infobox}

\section{React Bootstrap}

\href{https://react-bootstrap.github.io}{React Bootstrap} supplies us with pre-styled Bootstrap components, meaning that we don't need to worry about assigning \texttt{className}s.
\\

First, make sure that you're including the Bootstrap 4 CSS in your \texttt{index.html}:

\begin{minted}[fontsize=\scriptsize]{html}
    <link
        rel="stylesheet"
        href="https://maxcdn.bootstrapcdn.com/bootstrap/4.3.1/css/bootstrap.min.css"
        integrity="sha384-ggOyR0iXCbMQv3Xipma34MD+dH/1fQ784/j6cY/iJTQUOhcWr7x9JvoRxT2MZw1T"
        crossorigin="anonymous"
    />
\end{minted}

Now, install the \texttt{react-bootstrap} package:

\begin{minted}{bash}
    npm install react-bootstrap
\end{minted}

Let's create a simple button with a blue background. First, we need to import the relevant component from \texttt{react-bootstrap}:

\begin{minted}{jsx}
    import { Button } from "react-bootstrap";
\end{minted}

This component has a \texttt{variant} prop and whatever it is given as children appears as the button text:

\begin{minted}{jsx}
    <Button variant="primary">I'm Blue Da Ba De Da Ba Di</Button>
\end{minted}

This is why documentation is important: without the React Bootstrap documentation there's no way I could know how to use the \texttt{Button} component without looking at the source code (and even then it might not be obvious). I can't just guess that there's a \texttt{variant} prop and hope it works.
\\

React Bootstrap component have prop pass-through, meaning that any valid DOM props that you pass to a component will automatically get added to the underlying DOM element. For example, we could pass an \texttt{onClick} prop and this would get added as an \texttt{onClick} event-handler to the underlying HTML element.
\\

Using this, let's recreate the toggle buttons from when we looked at lifting state:

\inputminted{js}{06/figures/01/01-Buttons.js}

As you can see, it's very similar to before, except we need to import \texttt{Button} from \texttt{react-bootstrap} and use the correct props: in this case \texttt{variant} and \texttt{onClick} instead of \texttt{selected} and \texttt{handleClick}.
\\

This is just a single component from React Bootstrap. Read the \href{https://react-bootstrap.github.io/getting-started/introduction/}{React Bootstrap Documentation} to see all the other components and how to use them.

\begin{infobox}{React Native}
    \href{https://facebook.github.io/react-native/}{React Native} is a system that allows you to turn regular JavaScript React apps into ``native'' mobile apps for iPhone and Android.
    \\

    It does this by giving you a set of pre-built components, just like React Bootstrap. However, rather than representing HTML elements, React Native components represent native UI components: like buttons, sliders, etc.
\end{infobox}


\section{React Datepicker}

\href{https://github.com/Hacker0x01/react-datepicker/}{React Datepicker} is a single component, but it adds some very useful functionality.
\\

It's not uncommon on websites to have a date-picker for one reason or another. Some browsers support a built-in date-picker, but support is sporadic and inconsistent. Some of them don't look/work very nice either.
\\

Writing your own date-picker is \textit{a lot} of work: doing anything involving dates and times is always more complicated than you'd think. So it's not something you'd want to write yourself.
\\

Firstly, we need to import the React Datepicker CSS:

\begin{minted}{js}
    import "react-datepicker/dist/react-datepicker.css";
\end{minted}

Now we can use the date picker:

\inputminted{js}{06/figures/01/02-Dates.js}

As you can see, we've used the \texttt{selected} and \texttt{onChange} props of the \texttt{DatePicker} component. Again, we only know about these because of the documentation.
\\

If you have a look at the documentation you'll see that there are lots of other options that we can configure using props.

\hr

Pre-built components are a great way of adding functionality to your React app without having to write everything from scratch. If you're ever creating an app and you've got a bit of functionality that seems like it might be quite common, then have a look on NPM and see if someone's already written it for you.



\section{Additional Resources}

\begin{itemize}[leftmargin=*]
    \item \href{https://rsuitejs.com/en/}{React Suite}: A suite of React components
    \item \href{https://react-rainbow.web.app}{React Rainbow}: Another suite of React components
    \item \href{https://github.com/JoshK2/react-spinners-css}{React Spinners CSS Loaders}
    \item \href{https://github.com/atlassian/react-beautiful-dnd}{Drag and Drop List Items}
    \item \href{https://github.com/uiwjs/react-md-editor}{Markdown Editor}
    \item \href{https://catc.github.io/react-timekeeper/}{React Timekeeper}
\end{itemize}
