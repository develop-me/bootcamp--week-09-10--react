Hooks are a newer way of writing stateful components in a more functional style, without needing to use a \texttt{class}. This avoids issues with \texttt{this} and, done properly, can lead to more composable code.
\\

Let's take a look at a basic \texttt{class} based component again:

\js{}{13-hooks/figures/01-class-component}

And here it is using hooks:

\js{}{13-hooks/figures/02-hooks}

The hooks version is less than half the length, but it does exactly the same thing.
\\

Before we delve into exactly how this works, we'll need to look at destructuring arrays.

\section{Destructuring Assignment: Arrays}
So far we've only used destructuring to get properties from an object:

\begin{minted}{jsx}
    let person = { firstName: "Ben", lastName: "Folds", favouriteNo: 5 };

    // get the firstName property and store it in a firstName variable
    let { firstName } = person;
    console.log(firstName); // "Ben"
\end{minted}

We can also destructure an array. However, we don't have property names to work with and we're not allowed (and wouldn't want) variables that are just numbers. So destructuring an array is based purely on the \textit{order} of the items in the array:

\begin{minted}{jsx}
    let values = [1, 2, 3, 4];

    // get first and second value
    let [first, second] = values;
    console.log(second); // 2
\end{minted}

In the above example we could have called the variables \texttt{first} and \texttt{second} anything we liked: it's purely the order that's important. You'll also notice that we've ignored the third and fourth values in the array.


\section{\texttt{useState}}

Now we understand destructuring arrays, we can start to understand \texttt{useState}.
\\

\texttt{useState} sets up state for a \textit{single} value. It does three things for us:

\begin{itemize}
    \item Setting up the initial value
    \item Giving us the current value
    \item Giving us a function to update the value
\end{itemize}

In the counter example, we want to keep track of a value that we're calling \texttt{counter}. It should have an initial value of \texttt{0}, so we call \texttt{useState(0)} passing in the initial value.
\\

The \texttt{useState} function returns an array containing two values: the current value and a function to update that value. We can then destructure this array to get the value and update function back, using whatever names are most appropriate (usually using the \texttt{x} and \texttt{setX} convention):\footnote{If it returned an object it would be messier to name them whatever we like}

\begin{minted}{jsx}
    // useState returns an array containing two values
    // the current value
    // a function to update the value
    const [counter, setCounter] = useState(0);
\end{minted}

As mentioned above, \texttt{useState} can only be used for setting/updating a \texttt{single} value. You will need more than one call to \texttt{useState} if you need more than one value in state:

\js{}{13-hooks/figures/03-multiple-useState}

As you can imagine, this gets quite messy if you have lots of values in state. In that case it's best to use \texttt{useReducer}.

\pagebreak

\begin{infobox}{Tuples}
    You might notice that \texttt{useState} returns an array containing mixed data-types: any type for the first value and a function for the second. We've previously seen that mixing your types in an array is a \textbf{bad idea}, as generally you want to do the same thing to every item in an array. The exception to this is if you're using an array as a ``tuple''.
    \\

    Many programming languages, particularly functional ones, have a data type called a tuple. A tuple is a data-structure that contains two values that are somehow related. It might look something like this:

    \begin{minted}{haskell}
        (40.242531, -109.013390) -- a GPS coordinate
        ("Marte", 99) -- a name and age
    \end{minted}

    It's common, although not necessary, for the two parts of a tuple to be \textit{different} data types: it just ties two values together so they don't get separated.
    \\

    JavaScript doesn't have native support for tuples, so the closest approximation is to use an array with two values.
\end{infobox}


\section{\texttt{useReducer}}

\subsection{Reducers}

\textbf{Reducers} are like the \texttt{reduce} method of arrays, except that instead of accumulating a value by going through each item of an array, they accumulate the current value of state over time. Simples!
\\

A reducer is a \textbf{pure} function which gets passed the current value of the state plus some additional data, it then has to return a new version of the state based on those two bits of information. Reducers should be pure functions and \textbf{must always return a valid copy of the state}.
\\

The simplest possible reducer is one that takes the current state and returns it:

\js{}{13-hooks/figures/04-reducer-id}

This is not particularly useful, as it means the state can never change!
\\

For our reducer to be useful we need to also pass along some additional information, in React\footnote{And Redux} this is normally an ``action''.
\\

An action is just an object literal with various properties. If we give it a \texttt{type} property then we can change the state in different ways depending on the value of this specific property:

\js{}{13-hooks/figures/05-reducer-p1scores}

In the above reducer if the action has a \texttt{type} property of \texttt{PLAYER\_1\_SCORES}\footnote{It's a convention to use uppercase separated by underscores} we add \texttt{1} to the \texttt{player1} property of state. In all other cases (\texttt{default}) it will just return state unchanged - remember, \textbf{we always need to return a valid copy of the state}.
\\

We can now easily add a \texttt{PLAYER\_2\_SCORES} action:

\begin{minted}{jsx}
    case: "PLAYER_2_SCORES": return {
        ...state,
        player2: state.player2 + 1,
    };
\end{minted}

It's sometimes tidier to pull the \texttt{case}s out into their own functions:

\js{}{13-hooks/figures/06-reducer-pull}

If you do this make sure you pass \texttt{state} and, if necessary, \texttt{action} as arguments.


\subsection{Action Payloads}

Remember, a reducer is a pure function, so it should only work with information that's been passed in (or other pure functions).
\\

But because an action is just an object literal, we can have as many properties as we like on it.
\\

Say we wanted to keep track of lots of different player scores. It would be nice to be able to just have a single action and say \textit{which} player has scored:\footnote{This can be made even shorter using \href{https://tylermcginnis.com/computed-property-names/}{computed property names}}

\js{}{13-hooks/figures/07-reducer-payload}

This additional data is often referred to as the \textbf{payload}.


\subsection{Dispatching Actions}

Now that we understand what a reducer is, we need to know how to use it.
\\

We never call the reducer directly, instead we use the \texttt{dispatch} function that \texttt{useReducer} gives us:

\js{}{13-hooks/figures/08-useReducer}

We pass \texttt{useReducer} the reducer that we've created as well as an initial value for that state.\footnote{In the same way that we have to give an initial value for the accumulator in an array reduce} \texttt{useReducer} then gives us back two values: \texttt{state} and \texttt{dispatch}.
\\

The first value, which we've called \texttt{state}, is an object that represents the \textit{current} state (which will be \texttt{initial} the first time the component renders). We can destructure this to get the values that we're interested in.
\\

The second value, which we've called \texttt{dispatch}, is a function that we call, passing in an action as the first argument. Dispatching an action will cause React to call the reducer function, passing in the \textit{current} state along with the action that was dispatched. Once the reducer returns a version of the state, the component re-renders with the new state.
\\

It's usually always tidier to use \texttt{useReducer} when your state contains multiple values. Used properly, it can also make your code much more composable.

\begin{infobox}{Double Destructuring}
    In the above example we could have used multi-level destructuring to save a line of code:

    \begin{minted}[fontsize=\footnotesize]{javascript}
        // rather than destructuring state on a separate line
        // destructure state within the useReducer destructuring
        const [ { player1, player2 }, dispatch ] = useReducer(reducer, initial);
    \end{minted}

    You can see that we've destructured the state directly within the destructuring of the \texttt{useReducer} tuple.
\end{infobox}


\section{\texttt{useEffect}}

We need to be careful about where we put some of our code. We don't want to get in the way of React's ``render phase''.
\\

With our \texttt{class} based components we used the lifecycle methods (e.g. \\ \texttt{componentDidMount}) to avoid side-effects (changing the DOM, adding event listeners, \&c.) inside \texttt{render}, but with hooks based components we don't have methods and our function is effectively the \texttt{render} method.
\\

Anything that causes side-effects needs to be wrapped with \texttt{useEffect}: this delays running the code until \textit{after} the component has rendered. It can be used to recreate Lifecycle methods, although that is not its only purpose.
\\

A simple example might be if you wanted to updated the \texttt{<title>} tag of the page every time something changes on your page:

\js{}{13-hooks/figures/09-useEffect}

This function will be called \textit{every} time that component updates, similar to the behaviour of \texttt{componentDidUpdate}.
\\

Sometimes, if the component has more than one value in state, the update will not change the value that we're interested in. \texttt{useEffect} lets us optionally pass in an array of values that the effect is dependent on. Now the function will only run if the component updates \textit{and} that value has changed:

\js{}{13-hooks/figures/10-useEffect-dependents}

React only does a shallow check of the values in the array, so be careful passing in data structures, as it might not behave as you expect.
\\

If we want to recreate the behaviour of \texttt{componentDidMount} we can pass an \textit{empty array} as the second argument to \texttt{useEffect}. This will run the first time the component renders and then, because it only runs if the values in the given array change, but you've \textit{not passed in any values to check}, it won't run again:

\js{}{13-hooks/figures/11-StarWarsFolks}


We can also recreate the \texttt{addEventListener} example from the lifecycle methods section:


\js{}{13-hooks/figures/12-useEffect-return}

Notice that the \texttt{componentWillUnmount} code goes in a function which you return from the function passed to \texttt{useEffect}.\footnote{This is a ``higher-order function'': a function that accepts/returns another function} This is a nice way of doing it, as it groups the adding and removing of the event listener in one place.


\section{Custom Hooks}

We need to use \texttt{useEffect} within our component function because it uses the state of the component, so that needs to be in scope. However, this means that we can't reuse the logic inside \texttt{useEffect}, which restricts function reuse. Sad times.
\\

Say we needed various components that load all the articles from an API: an articles list for the homepage and an archive list for the sidebar. We can write our own custom hook to avoid needing to rewrite the article fetching code:

\js{}{13-hooks/figures/13-useGetArticles}

We create a function that \textit{just} deals with the state aspect of articles - no JSX involved. It's good practice to prefix the function name with \texttt{use}, to make it clear it's a custom hook. The function takes an initial value as its argument and returns a tuple containing the state and state setting function. We can now use this in any component we like:

\js{}{13-hooks/figures/14-Articles}

All we've really done is moved the article fetching code into a separate function, which we then call instead of \texttt{useState}. We don't even need to destructure the setter function as the custom hook is doing all the work for us.

\section{Rules of Hooks}

Hooks can only work if the order that they are called in never changes. Because of this, you cannot call hooks inside loops, conditions, or nested functions. They should \textbf{always be at the top level of your React function}.

\begin{infobox}{Under the Hook}
    Hooks seem almost magical. Where's the state actually getting stored? If the component function is called every time it renders, how is it being kept track of?
    \\

    How hooks actually work under the hood is a little bit messy. If you didn't know better, you might think that functions like \texttt{setState} are pure: but they keep track of values over time, and this is a give away that something very impure is going on under the hood.
    \\

    \textbf{You don't need to understand this}, but here is a super-simplified version of what's going on (it's waaay more complicated in real life):

    \js{}{13-hooks/figures/15-under-the-hood}

    As you can see, state is being kept track of outside of the component. And there's a bunch of other global(ish) variables too. Flagrantly impure.
    \\

    You can hopefully also see why the order that hooks are called in is so important: React \textit{only} has the order to work with.
\end{infobox}


\section{Additional Resources}

\begin{itemize}[leftmargin=*]
    \item \href{http://reactjs.org/docs/hooks-rules.html}{React: Rule of Hooks}
    \item \href{https://leewarrick.com/blog/react-use-effect-explained/}{React's useEffect and useRef Explained for Mortals}
    \item \href{https://tvernon.tech/blog/react-custom-hook-for-forms}{Writing your own React custom hooks}
    \item \href{https://overreacted.io/a-complete-guide-to-useeffect/}{A Complete Guide to \texttt{useEffect}}
    \item \href{https://medium.com/javascript-scene/do-react-hooks-replace-redux-210bab340672}{Do React Hooks Replace Redux?}
    \item \href{https://www.youtube.com/watch?v=nUzLlHFVXx0}{Composing Behavior in React or Why React Hooks are Awesome}
    \item \href{https://medium.com/@ntgard/tuples-in-javascript-cd33321e5277}{Tuples in JavaScript}
    \item \href{https://eloquentjavascript.net/05_higher_order.html}{Higher Order Functions}
    \item \href{https://kentcdodds.com/blog/usememo-and-usecallback/}{When to use \texttt{useMemo} and \texttt{useCallback}}
\end{itemize}
