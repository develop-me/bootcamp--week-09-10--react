\section{Routing}

\textbf{Routing} is the process of taking a URL and deciding what code should run.
\\

For example:

\begin{itemize}
    \item \texttt{app.dev/}: go to the homepage
    \item \texttt{app.dev/login}: show a login form
    \item \texttt{app.dev/register}: show a registration form
    \item \texttt{app.dev/posts}: show a list of all the posts
    \item \texttt{app.dev/posts/24}: show the post that has the \texttt{id} of 24
\end{itemize}

There are a lot of intricacies to writing routing code, so we're better to let someone else do it for us.
\\

That's where \textbf{ReactRouter} comes in.



\section{Setup}

First we need to install ReactRouter:

\begin{minted}{shell}
    npm install react-router-dom
\end{minted}

There are various versions of ReactRouter. We want the DOM version as we're building a single page web app.
\\

Then we need to update our \texttt{App} to use routing.
\\

First we need to import the ReactRouter components \texttt{<Router>} and \texttt{<Route>}. Then we need to wrap our existing \texttt{App} code with the \texttt{<Router>} component:

\js{}{07-router/figures/01-Router}

Now we can begin to add \textbf{routes}.


\pagebreak


\section{Routes}

\subsection{Basic Routes}

Visiting any URL will result in seeing the \texttt{<Header>} and \texttt{<Content>}.
\\

But, let's say we only wanted to show the \texttt{<Content>} element on the homepage. We could wrap it with a \texttt{<Route>}:

\js{}{07-router/figures/02-component}

Now if we visit the homepage of our app we'll see \texttt{<Content>}, but if we go to any other URL we'll only see the \texttt{<Header>}.
\\

Effectively a \texttt{<Route>} is like a conditional that says ``if the URL matches the given path, show the component''. Anything not in a \texttt{<Route>} component will \textit{always} render.


\subsection{Routes with Props}

How about if we wanted to show our \texttt{<Figure>} element if we visit \texttt{/figure}?
\\

A \texttt{<Figure>} component requires specific props in order to work, so we need to provide a bit more information.

\pagebreak

To do this we can wrap a component with a \texttt{<Route>}:

\begin{minted}{jsx}
    <Route path="/figure">
        <Figure
            caption="A cat, strutting its stuff!"
            src="https://goo.gl/tRdW93"
        />
    </Route>
\end{minted}

This lets us pass in a fully formed component \textit{including} props.



\section{Matches}

Say we had a list of articles and we wanted to show the user one specific article. On a standard website we might use a URL structure like the following:

\begin{itemize}
    \item \texttt{/articles/1}: show the article with id 1
    \item \texttt{/articles/2}: show the article with id 2
    \item \texttt{/articles/3}: show the article with id 3
    \item \&c.
\end{itemize}

We can reproduce this using ReactRouter's \texttt{render} prop and the \texttt{match} property:

\begin{minted}{jsx}
    <Route path="/articles/:id" render={ ({ match }) => (
        <Article article={ match.params.id } />
    ) } />
\end{minted}

The \texttt{:id} part of the URL is a \textbf{parameter}. We can call it whatever we like, as long as it starts with \texttt{:}. We can then access the value of a specific parameter by looking at the \texttt{match.params.<parameterName>} property that ReactRouter passes into the \texttt{render} function.\footnote{It also passes this as a prop to any components when we use the \texttt{component} prop on a \texttt{Route}.}


\pagebreak


For example, We can then use the passed value to show a particular article:

\js{}{07-router/figures/03-Article}



\section{Links}

We need a way of getting round our app. Normally we'd use an \texttt{<a>} tag, but that would cause the browser to load the page from scratch, and all our app's state would be lost.
\\

ReactRouter provides us with the \texttt{<Link>} component:

\js{}{07-router/figures/04-Link}

As long as we use \texttt{<Link>} components we'll never cause a hard-refresh, so our app's state will not be lost.
\\

Under the hood ReactRouter is creating an \texttt{<a>} element, adding an event handler with a call to \texttt{e.preventDefault()}, and then updating the address bar and browser history for us.



\section{404s}


We need some way to handle 404 errors.
\\

First, we'll need to create a \texttt{<FourOhFour>} component:

\begin{minted}{jsx}
    
    const FourOhFour = () => <p>Page not found</p>;

    export default FourOhFour;
\end{minted}


Next we will need the \texttt{<Switch>} component:

\begin{minted}{jsx}
    import {
        BrowserRouter as Router,
        Route,
        Switch,
    } from "react-router-dom";
\end{minted}

We can then wrap all our routes in the \texttt{<Switch>} component and add our \texttt{<FourOhFour>} component as the last item:

\begin{minted}{jsx}
    <Switch>
        <Route exact path="/">
            <Content />
        </Route>
        <FourOhFour />
    </Switch>
\end{minted}

Now, if none of the routes match it will display the last item in the \texttt{<Switch>} - like a \texttt{default} statement in a standard \texttt{switch()} conditional.
\\

Other than your 404 component, you should only wrap \texttt{<Route>} components with \texttt{<Switch>}, otherwise things won't display as you'd expect.


\section{Programmatic Navigation}

Sometimes it would be nice if our app could navigate for us. For example, when we submit a form. To do this we can use ReactRouter's \textbf{history} functionality.
\\

First, create a new file \texttt{src/history.js}:

\begin{minted}{js}
    import { createBrowserHistory } from "history";
    export default createBrowserHistory();
\end{minted}

Then add a \texttt{history} prop to your \texttt{<Router>}:

\begin{minted}{javascript}
    // import your history file
    import history from "../history";

    // use Router instead of BrowserRouter
    import { Router } from "react-router-dom";
\end{minted}

\begin{minted}{javascript}
    // pass history in as a prop to the Router
    <Router history={ history }>
\end{minted}

Now we can import this file and use it elsewhere to change the URL, for example in the \texttt{handleSumbit} method of a form component:

\begin{minted}{jsx}
    import history from "../history";
\end{minted}

\begin{minted}{js}
    handleSumbit(e) {
        // ...do form submission stuff

        // go to the homepage
        history.push("/");
    }
\end{minted}




\section{Additional Resources}

\begin{itemize}[leftmargin=*]
    \item \href{https://reacttraining.com/react-router/web/guides/philosophy}{React Router}
\end{itemize}
